\documentclass[aspectratio=169,pdf,hyperref={unicode},14pt]{beamer}

\usepackage[T2A]{fontenc}
\usepackage[utf8]{inputenc}
\usepackage[russian,english]{babel}
\usepackage{listings}
\usepackage{color}
\usepackage{default}
\usepackage{hyperref}

% Настройки бимера
\usetheme{Copenhagen}

% Настройки листинга
\definecolor{comments}{rgb}{0.5,0.5,0.5}
\definecolor{stdtypes}{rgb}{0.3,0.0,0.0}

\lstdefinestyle{cppcode}{
  belowcaptionskip=1\baselineskip,
  breaklines=true,
  frame=L,
  xleftmargin=\parindent,
  language=C++,
  showstringspaces=false,
  basicstyle=\footnotesize\ttfamily,
  morekeywords={constexpr,static_assert,decltype,co_await,co_return},
  keywordstyle=\bfseries\color{blue},
  commentstyle=\itshape\color{comments},
  stringstyle=\color{brown},
  classoffset=1,
  morekeywords=[1]{std,map,launch,future,shared_future,promise,packaged_task,async,move,to_string,when_any_result,when_any,tuple,when_all,list,vector,cout,flush},
  keywordstyle=[1]\bfseries\color{stdtypes},
}

\newcommand{\gooditem}[1]{\setbeamercolor{local structure}{fg=green}\item #1}
\newcommand{\baditem}[1]{\setbeamercolor{local structure}{fg=red}\item #1}
\newcommand{\isodoc}[1]{{\scriptsize \color{gray} #1}}

\title{Нестандартный future/promise}
\author{Сергей Видюк}
\date{15 февраля 2019}

\begin{document}

\begin{frame}
 \maketitle
\end{frame}

\begin{frame}[t]{Кратко обо мне}
 \begin{itemize}[<+->]
  \item Работаю в команде 3D карыт мобильной верси 2ГИС
  \item На последних трёх проектах был недоволен самописными API передачи данных между асинхронными задачами
  \item Написал свою реализацию \texttt{future/promise} API на базе Concurrency~TS \url{https://github.com/VestniK/portable_concurrency}
 \end{itemize}
\end{frame}

\begin{frame}[t]{Давайте напишем прогу которая...}
 \begin{itemize}[<+->]
  \item Рисует карту на базе web-тайлов
  \item Отображает большое количество маркеров разных приоритетов
  \item Выполняет разные задачи на разных потоках
  \item Не использует мутексов
  \item Не зависает и не падает на выходе
  \item Лежит на гитхабе: \url{https://github.com/VestniK/mapex}
 \end{itemize}
\end{frame}

\begin{frame}[fragile,t]{Загружаем тайлы}
  \begin{onlyenv}<1>
  \begin{lstlisting}[style=cppcode]
struct tile_id {
  int tx;
  int ty;
  int z_level;
};

class tile_source {
  QImage load(tile_id id);
};
  \end{lstlisting}
  \begin{itemize}
   \item Пусть синхронная функция загрузки у нас уже есть
  \end{itemize}
  \end{onlyenv}

  \begin{onlyenv}<2-3>
  \begin{lstlisting}[style=cppcode]
class tile_widget {
...
private:
  tile_source src_;
  map<tile_id, std::future<QImage>> tasks_;
};
...
tasks_[id] = std::async(
  launch::async,
  [=] {return src_.load(id);}
);
  \end{lstlisting}
  \begin{itemize}
   \begin{onlyenv}<2>
    \item Воспользуемся \texttt{std::async}
   \end{onlyenv}
   \begin{onlyenv}<3>
    \baditem Удаление задач только через ожидание их завершения
    \baditem На выходе ждём завершения всех текущих задач
   \end{onlyenv}
  \end{itemize}
  \end{onlyenv}

  \begin{onlyenv}<4>
  \begin{lstlisting}[style=cppcode]
class tile_widget {
...
private:
  tile_source src_;
  map<tile_id, pc::future<QImage>> tasks_;
};
...
tasks_[id] = pc::async(
  QThreadPool::globalInstance(),
  [=] {return src_.load(id);}
);
  \end{lstlisting}
  \begin{itemize}
   \item Перейдём на \texttt{pc::async}
  \end{itemize}
  \end{onlyenv}

\end{frame}

\begin{frame}{Спасибо за внимание}
\centerline{\includegraphics[height=4.5cm]{question_mark_blue.png}}
\end{frame}

\end{document}
